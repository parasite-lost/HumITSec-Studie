\documentclass[]{beamer}
%language
\usepackage[ngerman,english]{babel}
\usepackage[utf8]{inputenc}
\usepackage{xspace}
\usepackage{amsmath,amssymb}
\usepackage{ulem}

\usepackage{algpseudocode}
\usepackage{listings}

\usepackage{tikz}
\usetikzlibrary{mindmap}

\usepackage[utf8]{inputenc}
%theme
\useoutertheme[nofootline]{wuerzburg}
\useinnertheme{chamfered}
\usecolortheme{shark}

%title setup
\title{Generischer Titel: Studie - Emailkryptographie}
%\subtitle{Montag 12-14, Raum 01.150-128}
\subtitle{Human Factors in IT-Security}
\author{Ulrich Dorsch, Tim Grocki, Leonhard Hösch  \\ {\tiny ulrichdorsch@googlemail.com, tgrocki@lavabit.com, lh@dancingwolf.de \\}}
\institute[Universität Erlangen]{Friedrich-Alexander-Universität Erlangen
\\Department Informatik \\ Lehrstuhl 1}


\begin{document}
%titlepage
\begin{frame}
	\titlepage
\end{frame}

\begin{frame}
	\tableofcontents
\end{frame}

\section*{Überblick}
\begin{frame}
	%xkcd pgp
\end{frame}

\subsection*{Motivation}
\begin{frame}{Motivation}
\begin{itemize}
	\item Emailkryptographie ist nicht weit verbreitet.
	\item Wir möchten die Ursachen dafür herausfinden.
	\item Studien haben versucht zu zeigen, dass schlechte Usability der Grund f\"ur die geringe Verbreitung ist.
	\item Wir haben versucht, dies mit unserer Studie zu überprüfen.
\end{itemize}
\end{frame}


\subsection*{Forschungsfrage}
\begin{frame}{Forschungsfrage}
\begin{itemize}
\item Ist Usability das Hauptproblem der Emailverschl\"usselung?
\item Wir versuchten zu belegen, dass Usability nicht das primäre Problem
		hinter der geringen Verbreitung von Emailverschl\"usselung ist.
\end{itemize}
\end{frame}

\subsection*{Hypothese}
\frame[t]{
\begin{itemize}
    \item
	  Usability ist nicht das primäre Problem, da mehr als 50\% der
	  potenziellen Nutzer durch andere Ursachen davon abgehalten werden,
	  Emailkryptographie zu nutzen.
      \begin{itemize}
        \item Studie: möglichst umfassende Indentifikation dieser Ursachen
      \end{itemize}
\end{itemize}
\begin{block}{Usability Problem - Definition}
	In unserem Kontext betrachten wir ein Usability Problem als eine technische Hürde,
	die die Benutzung von Emailkryptographie verhindert oder deutlich erschwert.
\end{block}
}

\section*{Umfrageumstände}
%% NSA
\begin{frame}{NSA Skandal}
	bla
\end{frame}

\section*{Demographische Daten}
\begin{frame}{Demographische Daten}
	
\end{frame}

\section*{Auswertung der Antworten}
\subsection*{Interesse an Emailkryptographie}
\begin{frame}{Interesse an Emailkryptographie}
	
\end{frame}

\subsection*{Bekanntheit von Gefahren}
\begin{frame}{Bekanntheit von Gefahren}
	
\end{frame}

\subsection*{Schema der Fragen}
\begin{frame}{Fragebogenverlauf}
	\begin{tikzpicture}[mindmap,scale=0.7]
		\path[concept color=gray,every concept/.append style={scale=0.7}]
		node[concept] {255 \\ 100\%}
			child[concept color=gray,grow=-5] {node[concept] {N}}
			child[concept color=gray,grow=-170] {node[concept] {installiert}
				child[concept color=gray,grow=-30] {node[concept] {war inst.}
					child[concept color=gray,grow=-90] {node[concept] {delete}
						child[concept color=gray,grow=-130] {node[concept] {U}}
						child[concept color=gray,grow=-70] {node[concept] {U}}
					}
					child[concept color=gray,grow=-10] {node[concept] {geplant}
						child[concept color=gray,grow=0] {node[concept] {N}}
						child[concept color=gray,grow=-60] {node[concept] {versucht}
							child[concept color=gray,grow=-130] {node[concept] {U}}
							child[concept color=gray,grow=-50] {node[concept] {N}}
						}
					}
				}
				child[concept color=gray,grow=-130] {node[concept] {reglm}
					child[concept color=gray,grow=-130] {node[concept] {N}}
					child[concept color=gray,grow=-70] {node[concept] {jemals}
						child[concept color=gray,grow=-130] {node[concept] {U}}
						child[concept color=gray,grow=-70] {node[concept] {kontakt}
							child[concept color=gray,grow=-130] {node[concept] {U}}
							child[concept color=gray,grow=-70] {node[concept] {N}}
						}
					}
				}
			};
	\end{tikzpicture}
\end{frame}

\subsection*{Auswertung}
\begin{frame}{Auswertung}
	\begin{tikzpicture}[mindmap,scale=0.7]
		\path[concept color=gray,every concept/.append style={scale=0.7}]
		node[concept] {255 \\ 100\%}
			child[concept color=gray,grow=-5] {node[concept] {N}}
			child[concept color=gray,grow=-170] {node[concept] {installiert}
				child[concept color=gray,grow=-30] {node[concept] {war inst.}
					child[concept color=gray,grow=-90] {node[concept] {delete}
						child[concept color=gray,grow=-130] {node[concept] {U}}
						child[concept color=gray,grow=-70] {node[concept] {U}}
					}
					child[concept color=gray,grow=-10] {node[concept] {geplant}
						child[concept color=gray,grow=0] {node[concept] {N}}
						child[concept color=gray,grow=-60] {node[concept] {versucht}
							child[concept color=gray,grow=-130] {node[concept] {U}}
							child[concept color=gray,grow=-50] {node[concept] {N}}
						}
					}
				}
				child[concept color=gray,grow=-130] {node[concept] {reglm}
					child[concept color=gray,grow=-130] {node[concept] {N}}
					child[concept color=gray,grow=-70] {node[concept] {jemals}
						child[concept color=gray,grow=-130] {node[concept] {U}}
						child[concept color=gray,grow=-70] {node[concept] {kontakt}
							child[concept color=gray,grow=-130] {node[concept] {U}}
							child[concept color=gray,grow=-70] {node[concept] {N}}
						}
					}
				}
			};
	\end{tikzpicture}
\end{frame}
\subsection*{Details}
% detailfragen
\begin{frame}{Details}
	
\end{frame}

\section*{Forschungsergebnis}
\begin{frame}{Forschungsergebnis}
	
\end{frame}

\section*{Einschränkungen}
% meta probleme
% problem: knappe frage Emailkryptographie <-> unbekanter begriff
% hauptsächlich techniker/studenten
\begin{frame}{Einschränkungen}
	
\end{frame}


\end{document}
